%% ===== Begin LaTeX file ===========================
%%
\documentclass[10pt,a5paper,twoside]{article}
\usepackage{pandoc/coling2012}
\usepackage[frenchb]{babel}

\title{Interface Tangible comme Aide à la Maîtrise de l'Énergie}

\author{Maxime Daniel, Guillaume Rivière, Nadine Couture}


\begin{document}
\maketitle
\keywords{IHM, TUI, Persuasion Informatique, Espaces publics et physiques d'interaction sociale}

\section{Contexte}\label{contexte}

\begin{enumerate}
\def\labelenumi{\arabic{enumi}.}
\itemsep1pt\parskip0pt\parsep0pt
\item
  (?) Le Développement durable et ses enjeux (le transport, la gestion
  des déchets, la gestion de l'énergie, etc.).
\item
  (?) L'utilisation de l'Interaction Homme-Machine (IHM) pour aider à
  l'accomplissement du développement durable (l'étude des utilisateurs,
  la conception d'interaction durable, les technologies persuasives).
\item
  La transition énergétique et ses enjeux.
\item
  Des réseaux électriques classiques inadaptés.
\item
  (?) Une transition lente vers les réseaux électriques intelligents.
\item
  Le besoin de sensibiliser la population aux problématiques de gestion
  de l'énergie (consommation, production, stockage) pour ensuite
  persuader les individus à changer leurs comportements de consommation
  lorsque cela est nécessaire (consommer pendant les heures creuses,
  etc.).
\end{enumerate}

\section{Problématique}\label{probluxe9matique}

\subsection{Constat}\label{constat}

\begin{enumerate}
\def\labelenumi{\arabic{enumi}.}
\itemsep1pt\parskip0pt\parsep0pt
\item
  Pour persuader les individus à changer de comportement, beaucoup de
  travaux en persuasion informatique utilisent une stratégie de
  persuasion avec comme source principale de motivation, la réduction de
  la facture énergétique du domicile.
\item
  Les individus ne se sentent pas concernés par la réduction de la
  facture énergétique sur les espaces publics et physiques d'interaction
  sociale (une école, une entreprise, un hôpital, etc.). L'application
  de la persuasion informatique sur ces espaces est délaissée.
\item
  D'autres sources de motivation commencent à être utilisées telles que
  le plaisir avec l'utilisation de la Ludification, voir même du
  \emph{Serious Game}.
\item
  Les interfaces graphiques (GUI) sont majoritairement utilisées comme
  support à la persuasion informatique. Cependant, il existe d'autres
  types d'interfaces Homme-Machine telles que les Interfaces Utilisateur
  Tangibles qui pourraient se montrer plus adaptées au support de la
  persuasion informatique pour certaines situations telles que pour le
  support à la persuasion informatique sur les espaces publics et
  physiques d'interaction sociale.
\end{enumerate}

\subsection{Positionnement}\label{positionnement}

\begin{enumerate}
\def\labelenumi{\arabic{enumi}.}
\itemsep1pt\parskip0pt\parsep0pt
\item
  Besoin de définir une stratégie de persuasion adaptée aux espaces
  publics et physiques d'interaction sociale.
\item
  Besoin de valider la supériorité des TUIs sur les interfaces
  graphiques en terme de support à la persuasion informatique sur les
  espaces publics et physiques d'interaction sociale.
\end{enumerate}

\section{État de l'art}\label{uxe9tat-de-lart}

\subsection{Interface Utilisateur
Tangible}\label{interface-utilisateur-tangible}

\begin{enumerate}
\def\labelenumi{\arabic{enumi}.}
\itemsep1pt\parskip0pt\parsep0pt
\item
  Définition des Interfaces Utilisateur Tangibles.
\item
  (?) Genres et exemples :

  \begin{itemize}
  \itemsep1pt\parskip0pt\parsep0pt
  \item
    Interaction tangible sur table
  \item
    Interface Utilisateur Incarnée
  \item
    Réalité Augmenté Tangible
  \item
    Affichage Ambiant
  \end{itemize}
\item
  Applications et exemples :

  \begin{itemize}
  \itemsep1pt\parskip0pt\parsep0pt
  \item
    Communication sociale
  \item
    Apprentissage
  \item
    Divertissement
  \item
    Musique et Performance
  \item
    Planification et résolution de problème
  \item
    Programmation
  \item
    Visualisation d'information
  \end{itemize}
\end{enumerate}

\subsection{Persuasion Informatique}\label{persuasion-informatique}

\begin{enumerate}
\def\labelenumi{\arabic{enumi}.}
\itemsep1pt\parskip0pt\parsep0pt
\item
  Définition de la Persuasion Informatique.
\item
  Définition de la Ludification et du \emph{Serious Game}.
\item
  Définition des Systèmes Ludo-Persuasifs.
\item
  Applications et exemples :

  \begin{itemize}
  \itemsep1pt\parskip0pt\parsep0pt
  \item
    Santé et Exercices
  \item
    consommation et/ou comportement écologique
  \item
    Économie, commerce, marketing, sécurité, sûreté
  \item
    Divertissement
  \item
    Autres
  \end{itemize}
\end{enumerate}

\section{Les Interfaces Utilisateurs Tangibles comme support à la
Persuasion
Informatique}\label{les-interfaces-utilisateurs-tangibles-comme-support-uxe0-la-persuasion-informatique}

\subsection{Les principes de
Persuasion}\label{les-principes-de-persuasion}

\begin{enumerate}
\def\labelenumi{\arabic{enumi}.}
\itemsep1pt\parskip0pt\parsep0pt
\item
  La Grille de principes des Systèmes Ludo-Persuasifs (PISTIL, JIPS)
\item
  La Grille de caractérisation de la persuasion informatique (PISTIL,
  JIPS)
\end{enumerate}

\subsection{Les Forces des TUIs}\label{les-forces-des-tuis}

\begin{enumerate}
\def\labelenumi{\arabic{enumi}.}
\itemsep1pt\parskip0pt\parsep0pt
\item
  Représentation tangible
\item
  Incarnation Forte (Affordance et Iconicité)
\item
  Multiplexage de l'entrée de l'information dans l'espace et spontanéité
  de l'interaction
\item
  Collaboration
\item
  Réflexion tangible
\item
  Mobilité physique
\item
  Actions épistémiques et accessoire de réflexion
\item
  Applicabilité
\item
  Gestuelle
\end{enumerate}

\subsection{Persuasion Informatique
Ambiante}\label{persuasion-informatique-ambiante}

\begin{enumerate}
\def\labelenumi{\arabic{enumi}.}
\itemsep1pt\parskip0pt\parsep0pt
\item
  Définition de la Persuasion Informatique Ambiante.
\item
  Applications et Exemples :

  \begin{itemize}
  \itemsep1pt\parskip0pt\parsep0pt
  \item
    Santé et Exercices
  \item
    consommation et/ou comportement écologique (à détailler)
  \item
    Économie, commerce, marketing, sécurité, sûreté
  \item
    Divertissement
  \item
    Autres
  \end{itemize}
\end{enumerate}

\section{Caractérisation de la Persuasion
Informatique}\label{caractuxe9risation-de-la-persuasion-informatique}

\section{Caractérisation de la Persuasion
Informatique}\label{caractuxe9risation-de-la-persuasion-informatique-1}

\renewcommand\refname{Caractérisation de la Persuasion Informatique}
\bibliography{resources/biblio}


\end{document}

