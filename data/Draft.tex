% Some classes load the `subfigure` package which clashes with
% our internal use of `subfig` for subfloats. We are most likely
% not going to need the canned subfigure functionality anyways,
% so we'll trick LaTeX into thinking it already loaded `subfigure`
\makeatletter
\newcommand{\dontusepackage}[2][]{%
  \@namedef{ver@#2.sty}{9999/12/31}%
  \@namedef{opt@#2.sty}{#1}}
\makeatother
\dontusepackage{subfigure}


%% ===== Begin LaTeX file ===========================
%%
\documentclass[]{pandoc/sigchi}

\usepackage{lmodern}
\usepackage{amssymb,amsmath}
\usepackage{ifxetex,ifluatex}
\usepackage[usenames,dvipsnames]{color}
\usepackage{fixltx2e} % provides \textsubscript
\ifnum 0\ifxetex 1\fi\ifluatex 1\fi=0 % if pdftex
  \usepackage[T1]{fontenc}
  \usepackage[utf8]{inputenc}
\else % if luatex or xelatex
  \ifxetex
    \usepackage{mathspec}
    \usepackage{xltxtra,xunicode}
  \else
    \usepackage{fontspec}
  \fi
  \defaultfontfeatures{Mapping=tex-text,Scale=MatchLowercase}
  \newcommand{\euro}{€}
\fi
% use upquote if available, for straight quotes in verbatim environments
\IfFileExists{upquote.sty}{\usepackage{upquote}}{}
% use microtype if available
\IfFileExists{microtype.sty}{%
\usepackage{microtype}
\UseMicrotypeSet[protrusion]{basicmath} % disable protrusion for tt fonts
}{}
% disbable natbib, sigchi template handles that by itself
%\usepackage[]{natbib}
\bibliographystyle{apalike}
\usepackage{listings}
% Define slightly more reasonable Listings defaults
\lstset{
    basicstyle=\ttfamily\small,
    breaklines=true,
    prebreak=\raisebox{0ex}[0ex][0ex]{\ensuremath{\hookleftarrow}},
    frame=lines,
    showtabs=false,
    showspaces=false,
    showstringspaces=false,
    keywordstyle=\color[gray]{0.4}\bfseries,
    commentstyle=\color[gray]{0.65}\itshape,
    numbers=left,
    captionpos=b,
}
\ifxetex
  \usepackage[setpagesize=false, % page size defined by xetex
              unicode=false, % unicode breaks when used with xetex
              xetex]{hyperref}
\else
  \usepackage[unicode=true]{hyperref}
\fi
\hypersetup{breaklinks=true,
            bookmarks=true,
            pdfauthor={Maxime DANIEL; Guillaume RIVIERE; Nadine COUTURE},
            pdftitle={Tangible User Interface as support for Energy management},
            colorlinks=true,
            citecolor=black,
            urlcolor=blue,
            linkcolor=black,
            pdfborder={0 0 0}}
\urlstyle{same}  % don't use monospace font for urls
\setlength{\parindent}{0pt}
\setlength{\parskip}{6pt plus 2pt minus 1pt}
\setlength{\emergencystretch}{3em}  % prevent overfull lines
\setcounter{secnumdepth}{-2}

%% from sigchi
% useful for balancing the last columns
\usepackage{balance}


\title{Tangible User Interface as support for Energy management}
\author{Maxime DANIEL \and Guillaume RIVIERE \and Nadine COUTURE}
\date{}


% comment if you want LaTeX's default font
\usepackage{times}

% we'll override the "\caption*" that scholdoc uses with default figure markup
\usepackage{suffix}

% explicitely load subfig 'cause we won't use scholdoc touchy subfigures
\usepackage{subfig}


\begin{document}

% need a loop 'cause there is subfields

\maketitle
\begin{abstract}
TO DO.
\end{abstract}

\keywords{
      HCI;
      TUI;
      Persuasive technology;
      Public physical social spaces}

      \category{TO DO}{TO DO}{TO DO}
  

% from natbib commands to apalike + make it work in captions
\def \citep {\protect\cite}

%% From demo file

% Arabic page numbers for submission. 
% Remove this line to eliminate page numbers for the camera ready copy
\pagenumbering{arabic}

% llt: Define a global style for URLs, rather that the default one
\makeatletter
\def\url@leostyle{%
  \@ifundefined{selectfont}{\def\UrlFont{\sf}}{\def\UrlFont{\small\bf\ttfamily}}}
\makeatother
\urlstyle{leo}

% To make various LaTeX processors do the right thing with page size.
\def\pprw{8.5in}
\def\pprh{11in}
\special{papersize=\pprw,\pprh}
\setlength{\paperwidth}{\pprw}
\setlength{\paperheight}{\pprh}
\setlength{\pdfpagewidth}{\pprw}
\setlength{\pdfpageheight}{\pprh}

% Make sure hyperref comes last of your loaded packages, to give it a
% fighting chance of not being over-written, since its job is to
% redefine many LaTeX commands.
\definecolor{linkColor}{RGB}{6,125,233}
\hypersetup{%
  bookmarksnumbered,
  colorlinks,
  citecolor=black,
  filecolor=black,
  linkcolor=black,
  urlcolor=linkColor,
  breaklinks=true,
}

% we'll override the "\caption*" that scholdoc uses with default figure markup
\WithSuffix\newcommand\caption*{\caption}% *-variant

% In case we got a problem with scholdoc headers' levels
% from http://tex.stackexchange.com/a/61803
\newcommand{\leveldown}% Demote sectional commands
  {\let\section\subsection%
   \let\subsection\subsubsection%
   \let\subsubsection\paragraph%
   \let\paragraph\subparagraph%
   %\let\subparagraph\relax%
  }
  
\newcommand{\levelup}% Promote sectional commands
  {\let\subparagraph\paragraph%
   \let\paragraph\subsubsection%
   \let\subsubsection\subsection%
   \let\subsection\section%
   %\let\section\relax%
  }

% auto margin with subfigures
% FIXME: slight shift toward right...
\let\oldsubfloat\subfloat
\renewcommand*{\subfloat}{\hfill\oldsubfloat}


% promote all sections
\levelup

\section{Contexte}\label{contexte}

\begin{enumerate}
\def\labelenumi{\arabic{enumi}.}
\itemsep1pt\parskip0pt\parsep0pt
\item
  Le Développement durable et ses enjeux (le transport, la gestion des
  déchets, la gestion de l'énergie, etc.).
\item
  L'utilisation de l'Interaction Homme-Machine (IHM) pour aider à
  l'instauration du développement durable (l'étude des utilisateurs, la
  conception d'interaction durable, les technologies persuasives).
\item
  La transition énergétique et ses enjeux.
\item
  Des réseaux électriques classiques inadaptés.
\item
  (?) Une transition lente vers réseaux électriques intelligents.
\end{enumerate}

\section{Problematique}\label{problematique}

\begin{enumerate}
\def\labelenumi{\arabic{enumi}.}
\itemsep1pt\parskip0pt\parsep0pt
\item
  Le besoin de sensibiliser la population aux problématiques de gestion
  de l'énergie (consommation, production, stockage).
\item
  La nécessité de persuader les individus à changer leurs comportements
  de consommation lorsque cela est nécessaire (consommer pendant les
  heures creuses, etc.).
\item
  Beaucoup de travaux en persuasion informatique utilisent une stratégie
  de persuasion construite sur la réduction de la facture énergétique
  pour des espaces domestiques et physiques d'interaction sociale.
\item
  Les individus ne se sentent pas concernés par la réduction de la
  facture énergétique sur les espaces publics et physiques d'interaction
  sociale. L'application de la persuasion informatique sur ces espaces
  est délaissé.
\item
  Ma (première) position, appliquer la persuasion informatique sur les
  espaces publics et physiques d'interaction sociale et utiliser une
  stratégie de persuasion adaptée à ces espaces.
\item
  Les interfaces graphiques sont majoritairement utilisées comme support
  à la persuasion informatique. Cependant, il existe d'autres types
  d'interfaces Homme-Machine (e.g., TUI) qui pourraient être plus
  adaptés au support de la persuasion informatique sur les espaces
  publics et physiques d'interaction sociale.
\item
  Ma (seconde) position, utiliser les TUIs comme support à la persuasion
  informatique sur les espaces publics et physiques d'interaction
  sociale.
\end{enumerate}

\section{Etat de l'art}\label{etat-de-lart}

\subsubsection{Interface Utilisateur
Tangible}\label{interface-utilisateur-tangible}

\begin{enumerate}
\def\labelenumi{\arabic{enumi}.}
\itemsep1pt\parskip0pt\parsep0pt
\item
  Définition.
\item
  Interaction tangible sur table + exemple.
\item
  Interface Utilisateur Incarnée + exemple.
\item
  Réalité Augmenté Tangible + exemple.
\item
  Affichage Ambiant + exemple.
\end{enumerate}

\subsubsection{Persuasion Informatique, Ludification et Système
Ludo-Persuasif}\label{persuasion-informatique-ludification-et-systuxe8me-ludo-persuasif}

\begin{enumerate}
\def\labelenumi{\arabic{enumi}.}
\itemsep1pt\parskip0pt\parsep0pt
\item
  Définition de la Persuasion Informatique.
\item
  Modèle de la Persuasion Informatique.
\item
  Exemple.
\item
  Définition de la Ludification.
\item
  Modèle de la Ludification.
\item
  Exemple.
\item
  Définition des Systèmes Ludo-Persuasifs.
\item
  Modèle des Systèmes Ludo-Persuasifs.
\item
  Exemple.
\end{enumerate}

\subsubsection{Persuasion Informatique
Ambiante}\label{persuasion-informatique-ambiante}

\begin{enumerate}
\def\labelenumi{\arabic{enumi}.}
\itemsep1pt\parskip0pt\parsep0pt
\item
  Définition.
\item
  Modèle.
\item
  Applications.
\end{enumerate}

\section{Interface Utilisateur Tangible et Persuasion Informatique,
Naissance de la Persuasion Informatique Tangible un sous domaine de la
Persuasion Informatique (Soyons fou
!)}\label{interface-utilisateur-tangible-et-persuasion-informatique-naissance-de-la-persuasion-informatique-tangible-un-sous-domaine-de-la-persuasion-informatique-soyons-fou}

\begin{enumerate}
\def\labelenumi{\arabic{enumi}.}
\itemsep1pt\parskip0pt\parsep0pt
\item
  Les forces, et les faiblesses des Interfaces Utilisateur Tangibles.
\item
  Les forces, et les faiblesses de la Persuasion Informatique actuelle.
\item
  Justification de l'application des Interface Utilisateur Tangible à la
  Persuasion Informatique.
\end{enumerate}

\section{Caractérisation de la Persuasion Informatique
Tangible}\label{caractuxe9risation-de-la-persuasion-informatique-tangible}

\begin{enumerate}
\def\labelenumi{\arabic{enumi}.}
\itemsep1pt\parskip0pt\parsep0pt
\item
  Des tableaux de caractérisation des TUIs.
\item
  Des tableaux de caractérisation de la persuasion informatique et de
  ludification.
\item
  A partir de ces tableaux, proposition d'un tableau de caractérisation
  pour la Persuasion Informatique Tangible.
\end{enumerate}

\section{Un dispositif de Persuasion Informatique
Tangible}\label{un-dispositif-de-persuasion-informatique-tangible}

\subsubsection{Prototype}\label{prototype}

\begin{enumerate}
\def\labelenumi{\arabic{enumi}.}
\itemsep1pt\parskip0pt\parsep0pt
\item
  Prototypage pour la validation de l'application des TUIs à la
  persuasion informatique (valider la collaboration, la réflexion
  tangible, la représentation tangible, les actions épistémiques,
  affordance, etc.)
\end{enumerate}

\subsubsection{Experience}\label{experience}

\subsubsection{Resultats}\label{resultats}

% to put before content of last page
\balance{}

\renewcommand\refname{\ldots{}}
\bibliography{resources/biblio}


\end{document}

