%% ===== Begin LaTeX file ===========================
%%
\documentclass[10pt,a5paper,twoside]{article}
\usepackage{pandoc/coling2012}
\sloppy
\hyphenpenalty 10000
\title{Interface Tangible comme Aide à la Maîtrise de l'Énergie}

\author{Maxime Daniel, Guillaume Rivière, Nadine Couture}


\begin{document}
\maketitle
\keywords{IHM, TUI, persuasion informatique, espaces publics et physiques d'interaction sociale}

\section{Contexte}\label{contexte}

\begin{enumerate}
\def\labelenumi{\arabic{enumi}.}
\itemsep1pt\parskip0pt\parsep0pt
\item
  Le Développement durable et ses enjeux (le transport, la gestion des
  déchets, la gestion de l'énergie, etc.).
\item
  L'utilisation de l'IHM pour aider à l'accomplissement du développement
  durable (le développement durable dans la conception, le développement
  durable par la conception, etc.).
\item
  La transition énergétique et ses enjeux.
\item
  Des réseaux électriques classiques inadaptés.
\item
  (?) Une transition lente vers les réseaux électriques intelligents.
\item
  Le besoin de sensibiliser la population aux problématiques de gestion
  de l'énergie (consommation, production, stockage) et de persuader les
  individus à changer leurs comportements de consommation lorsque cela
  est nécessaire (consommer pendant les heures creuses, etc.).
\end{enumerate}

\subsection{Constats}\label{constats}

\begin{enumerate}
\def\labelenumi{\arabic{enumi}.}
\itemsep1pt\parskip0pt\parsep0pt
\item
  Pour persuader les individus à changer de comportement, beaucoup de
  travaux en persuasion informatique utilisent une stratégie de
  persuasion avec comme source principale de motivation, la réduction de
  la facture énergétique du domicile.
\item
  Les individus ne se sentent pas concernés par la réduction de la
  facture énergétique sur les espaces publics et physiques d'interaction
  sociale (une école, une entreprise, un hôpital, etc.). L'application
  de la persuasion informatique sur ces espaces est quelque peu
  délaissée.
\item
  D'autres sources de motivation commencent à être utilisées telles que
  le plaisir avec l'utilisation de la ludification, voir même du
  \emph{Serious Game}.
\item
  Les interfaces graphiques (GUI) sont majoritairement utilisées comme
  support à la persuasion informatique. Cependant, il existe d'autres
  types d'interfaces homme-machine telles que les interfaces utilisateur
  tangibles qui pourraient se montrer plus adaptées au support de la
  persuasion informatique pour certaines situations telles que pour le
  support à la persuasion informatique sur les espaces publics et
  physiques d'interaction sociale.
\end{enumerate}

\subsection{Positionnement}\label{positionnement}

\begin{enumerate}
\def\labelenumi{\arabic{enumi}.}
\itemsep1pt\parskip0pt\parsep0pt
\item
  Définir une stratégie de persuasion adaptée aux espaces publics et
  physiques d'interaction sociale.
\item
  Définir et valider les supériorités des TUIs sur les interfaces
  graphiques en terme de support à la persuasion informatique sur les
  espaces publics et physiques d'interaction sociale, voir plus
  généralement en terme de support à la persuasion informatique.
\end{enumerate}

\section{État de l'art}\label{uxe9tat-de-lart}

\begin{enumerate}
\def\labelenumi{\arabic{enumi}.}
\itemsep1pt\parskip0pt\parsep0pt
\item
  Balayage de la persuasion informatique.
\item
  Focus sur la persuasion informatique dédiée à la gestion de l'énergie.
\item
  Balayage des TUIs.
\item
  Balayage sur la persuasion informatique ambiante.
\item
  Focus sur la persuasion informatique ambiante dédiée à la gestion de
  l'énergie.
\end{enumerate}

\subsection{Persuasion informatique}\label{persuasion-informatique}

\begin{enumerate}
\def\labelenumi{\arabic{enumi}.}
\itemsep1pt\parskip0pt\parsep0pt
\item
  Définition de la persuasion informatique.
\item
  Définition de la ludification et du \emph{Serious Game}.
\item
  Définition des systèmes ludo-persuasifs.
\item
  Applications et exemples :

  \begin{itemize}
  \itemsep1pt\parskip0pt\parsep0pt
  \item
    Santé et Exercices.
  \item
    Économie, commerce, marketing, sécurité, sûreté.
  \item
    Divertissement.
  \item
    Autres.
  \item
    Consommation et/ou comportement écologique.
  \end{itemize}
\end{enumerate}

\subsubsection{Persuasion informatique dédiée à la gestion de
l'énergie}\label{persuasion-informatique-duxe9diuxe9e-uxe0-la-gestion-de-luxe9nergie}

\begin{enumerate}
\def\labelenumi{\arabic{enumi}.}
\itemsep1pt\parskip0pt\parsep0pt
\item
  Exemples.
\end{enumerate}

\subsection{Interface utilisateur
tangible}\label{interface-utilisateur-tangible}

\begin{enumerate}
\def\labelenumi{\arabic{enumi}.}
\itemsep1pt\parskip0pt\parsep0pt
\item
  Définition des TUIs.
\item
  Genres et exemples :

  \begin{itemize}
  \itemsep1pt\parskip0pt\parsep0pt
  \item
    Interaction tangible sur table.
  \item
    Interface Utilisateur Incarnée.
  \item
    Réalité Augmenté Tangible.
  \item
    Affichage Ambiant.
  \end{itemize}
\item
  Applications et exemples :

  \begin{itemize}
  \itemsep1pt\parskip0pt\parsep0pt
  \item
    Communication sociale.
  \item
    Apprentissage.
  \item
    Divertissement.
  \item
    Musique et Performance.
  \item
    Planification et résolution de problème.
  \item
    Programmation.
  \item
    Visualisation d'information.
  \end{itemize}
\end{enumerate}

\subsection{Persuasion informatique
ambiante}\label{persuasion-informatique-ambiante}

\begin{enumerate}
\def\labelenumi{\arabic{enumi}.}
\itemsep1pt\parskip0pt\parsep0pt
\item
  Définition de la persuasion informatique ambiante.
\item
  Applications et exemples :

  \begin{itemize}
  \itemsep1pt\parskip0pt\parsep0pt
  \item
    Santé et Exercices.
  \item
    Économie, commerce, marketing, sécurité, sûreté.
  \item
    Divertissement.
  \item
    Autres.
  \item
    Consommation et/ou comportement écologique.
  \end{itemize}
\end{enumerate}

\subsubsection{Persuasion informatique ambiante dédiée à la gestion de
l'énergie.}\label{persuasion-informatique-ambiante-duxe9diuxe9e-uxe0-la-gestion-de-luxe9nergie.}

\begin{enumerate}
\def\labelenumi{\arabic{enumi}.}
\itemsep1pt\parskip0pt\parsep0pt
\item
  Exemples.
\end{enumerate}

\section{Les interfaces utilisateurs tangibles comme support à la
Persuasion
Informatique}\label{les-interfaces-utilisateurs-tangibles-comme-support-uxe0-la-persuasion-informatique}

\begin{enumerate}
\def\labelenumi{\arabic{enumi}.}
\itemsep1pt\parskip0pt\parsep0pt
\item
  Définition des forces des TUIs.
\item
  Définition des principes de persuasion.
\item
  Pour chacune de ces forces (ou quelques unes), définir une hypothèse
  sur leur potentiel à améliorer le support à un ou plusieurs principes
  de persuasion.
\item
  Pour chacune des hypothèses, les valider : Prototyper, Évaluer,
  Analyser les résultats.
\end{enumerate}

\subsection{Les Forces des TUIs}\label{les-forces-des-tuis}

\begin{enumerate}
\def\labelenumi{\arabic{enumi}.}
\itemsep1pt\parskip0pt\parsep0pt
\item
  Collaboration (e.g., Expert de l'énergie/Particulier).
\item
  Applicabilité (e.g., contexte précis).
\item
  Réflexion tangible (i.e., aider à la compréhension, Renforcer la
  connexion entre le corps et la cognition).
\item
  gestuelle (i.e., alléger la charge cognitive).
\item
  Actions épistémiques et accessoires de réflexion ( i.e., offrir de la
  mémorisation externe).
\item
  Représentation tangible (i.e., guider, contraindre et déterminer le
  comportement cognitif).
\item
  Multiplexage de l'entrée de l'information dans l'espace et spontanéité
  de l'interaction (i.e., actions parallèles, prendre avantage de la
  forme, de la taille, et de la position des objets.)
\item
  Incarnation forte (i.e., créer de l'affordance et de l'iconicité).
\end{enumerate}

\subsection{Les principes de
persuasion}\label{les-principes-de-persuasion}

\begin{enumerate}
\def\labelenumi{\arabic{enumi}.}
\itemsep1pt\parskip0pt\parsep0pt
\item
  La Grille de principes des systèmes ludo-persuasifs.
\end{enumerate}

\ldots{}

\newpage
\bibliographystyle{apalike-fr}
\bibliography{resources/biblio}


\end{document}

