%% ===== Begin LaTeX file ===========================
%%
\documentclass[10pt,a5paper,twoside]{article}
\usepackage{pandoc/coling2012}
\usepackage{color}
\sloppy
\hyphenpenalty 10000
\title{Interface Tangible comme Aide à la Maîtrise de l'Énergie}

\author{Maxime DANIEL\\ Guillaume RIVIÈRE\\ Nadine COUTURE}
\date{\today}


\begin{document}

\maketitle
\keywords{IHM, TUI, technologie persuasive, espace publique physique, gestion de l'énergie}

\clearpage
\tableofcontents
\clearpage
\section{Contexte}\label{contexte}

\subsection{Le développement durable}\label{le-duxe9veloppement-durable}

Le développement durable est le développement qui subvient aux besoins
du présent sans compromettre la capacité des générations futures à
répondre à leurs propres besoins \citep{Hariem1985world}. Face à la
crise écologique et sociale à la quelle le monde fait face (changement
climatique, raréfaction des ressources naturelles, pénuries d'eau douce,
rapprochement du pic pétrolier, écarts entre pays développés et pays en
développement, sécurité alimentaire, déforestation et perte drastique de
biodiversité, croissance de la population mondiale, catastrophes
naturelles et industrielles) atteindre le développement durable est une
priorité indiscutable. Une grande majorité, si ce n'est pas la totalité,
des secteurs d'activités sont concernés par le développement durable.

\subsection{L'énergie}\label{luxe9nergie}

Le secteur de l'énergie est un acteur majeur du développement durable
depuis les premières crises énergétiques des années 1970. En 1973, 86.7
\% de la production mondiale d'énergie primaire provenait des
combustibles fossiles \citep{iea2015key}. À cette époque le système de
production énergétique était presque exclusivement basé sur l'énergie
fossile. Afin d'obtenir ce type d'énergie, il est nécessaire d'exploiter
des combustibles fossiles telles que le pétrole, le charbon ou le gaz
naturel qui sont disponibles en quantité limitée. En 1973, avec une part
de 46.2 \% \citep{iea2015key}, le pétrole était le combustible fossile
le plus utilisée dans le monde et à l'arrivé du premier choc pétrolier
de cette même année, le monde a fait l'expérience d'une pénurie de
pétrole, sa source d'énergie fossile principale. En plus du problème des
réserves limitées des combustibles fossiles, l'exploitation de ces
combustibles produit une grande quantité de polluant et contribue en
grande partie au réchauffement climatique. À la suite de ces évènements,
le secteur de l'énergie à entrepris la recherche d'énergies alternatives
à l'énergie fossile afin de réduire la dépendance aux combustibles
fossiles disponibles en quantité limité et réduire l'impact
environnementale (e.g., pollution, réchauffement climatique) de la
production de cette énergie. L'énergie nucléaire et l'énergie
renouvelable sont deux énergies alternatives à l'énergie fossile issues
de cette recherche.

L'énergie renouvelable est une énergie produisant pas ou très peu de
polluant et dont les ressources sont considérées comme inépuisables. Il
existe différents types d'énergie renouvelable telles que l'énergie
solaire, l'énergie éolienne, l'énergie hydraulique, la biomasse ou
encore l'énergie géothermique. En 2013, l'énergie primaire mondiale
produite était à 13.8 \% de l'énergie renouvelable, 4.8 \% de l'énergie
nucléaire et à 81.4 \% de l'énergie fossile \citep{iea2015key}.
L'énergie renouvelable fait partie du paysage énergétique mondiale et
forme avec l'énergie fossile et l'énergie nucléaire le mix énergétique.

En 40 ans, la part de l'énergie renouvelable dans le mix énergétique est
passé de 12.4 \% en 1973 à 13.8 \% en 2013 \citep{iea2015key}. Cette
lenteur dans la transition vers les énergies renouvelables, est la
conséquence d'une continuité dans l'exploitation des combustibles
fossiles dont l'économie restent, encore aujourd'hui, considérables mais
pas seulement. Les réseaux électriques classiques sont également
responsables de cette lenteur dans la transition énergétique vers le
renouvelable. Ils se montrent particulièrement inadaptés à l'intégration
des énergies renouvelables au paysage énergétique. Cette inadaptabilité
s'expliquer par la variabilité de certaines énergies renouvelables
(solaire, éolien, etc.) qui pose de réels problèmes de gestion de la
production d'électricité ou encore par la multiplication des sites de
production d'électricité (i.e.~décentralisation de la production
d'électricité) qui n'est pas adapté aux réseaux électriques classiques
qui sont conçus pour acheminer et non pour collecter l'électricité.
\textcolor{red}{à continuer vers smart grid et le besoin d'engager les consommateurs + références}

\subsection{L'interaction
homme-machine}\label{linteraction-homme-machine}

\citet{Blevis2007sustainable} évoque le besoin de changer le rôle joué
par l'IHM dans les cycles rapides d'obsolescence des produits qui
contribuent entre autres à la raréfaction des ressources naturelles et à
la pollution. Il expose la possibilité de réduire l'impact matériel de
la technologie à la fois directement (e.g., par la création de produit
qui peuvent être remplacer partiellement plutôt que complètement) et
indirectement (e.g., par la création de produits à qualité héréditaire
afin qu'ils ne soient pas rapidement remplacés).
\citet{mankoff2007environmental} offre une catégorisation de l'IHM pour
le développement durable en deux catégories : le développement durable
\emph{dans} la conception (mitiger l'impact matériel du
logiciel/matériel) et le développement durable \emph{par} la conception
(influencer les styles de vie et les prises de décision durables).
\citet{Reitberger2008surrounded} et \citet{Tscheligi2007persuasion}
affirment que la technologie persuasive peut être un ingrédient clé de
l'IHM pour le développement durable en informant les utilisateurs sur
l'impact environnemental de leurs actions et en augmentant la
désirabilité des comportements pro-environnementaux.
\textcolor{red}{à continuer plus en détails vers les technologies persuasives + références}

\section{Problématique}\label{probluxe9matique}

\begin{enumerate}
\def\labelenumi{\arabic{enumi}.}
\itemsep1pt\parskip0pt\parsep0pt
\item
  Pour persuader les individus à changer de comportement, beaucoup de
  travaux en technologie persuasive utilisent une stratégie de
  persuasion avec comme source principale de motivation, la réduction de
  la facture énergétique du domicile.
\item
  Les individus ne se sentent pas concernés par la réduction de la
  facture énergétique sur les espaces publiques physiques (une école,
  une entreprise, un hôpital, etc.) ; l'application de la technologie
  persuasive sur ces espaces est quelque peu délaissée.
\item
  D'autres sources de motivation commencent à être utilisées telles que
  le plaisir avec l'utilisation de la ludification, voir même du
  \emph{Serious Game}.
\item
  Les interfaces graphiques (GUI) sont majoritairement utilisées comme
  support aux technologies persuasives. Cependant, il existe d'autres
  types d'interfaces homme-machine telles que les interfaces utilisateur
  tangibles qui pourraient se montrer plus adaptées au support des
  technologies persuasives que les GUIs pour certaines situations (e.g.,
  support aux technologies persuasives sur les espaces publiques
  physiques).
\end{enumerate}

\section{État de l'art}\label{uxe9tat-de-lart}

\subsection{Technologie persuasive}\label{technologie-persuasive}

\begin{enumerate}
\def\labelenumi{\arabic{enumi}.}
\itemsep1pt\parskip0pt\parsep0pt
\item
  Définition de la technologie persuasive par \citet{fogg1998captology},
  \citet{fogg2002persuasive}.
\item
  Définition du \emph{Serious Game} par \citet{abt1970serious},
  \citet{ritterfeld2009serious} et de la ludification par
  \citet{deterding2011game}.
\item
  Définition des systèmes ludo-persuasifs par
  \citet{senach2015systemes}.
\item
  Applications :

  \begin{itemize}
  \itemsep1pt\parskip0pt\parsep0pt
  \item
    Santé - \textcolor{red}{à lire} \citet{bhatnagar2012biometric},
    \citet{chiu2009playful}, \citet{fabri2013changing},
    \citet{gasca2008persuasive}, \citet{halan2010high},
    \citet{kehr2012transformational}, \citet{kroes2013empowering},
    \citet{lee2011mining}, \citet{looije2006incorporating},
    \citet{nakajima2013designing}, \citet{parmar2008persuasive},
    \citet{salam2010using}, \citet{vanleer2012use} -.
  \item
    Exercices -\textcolor{red}{à lire} \citet{arteaga2010mobile},
    \citet{berkovsky2012physical}, \citet{consolvo2008flowers},
    \citet{consolvo2008activity}, \citet{foster2010motivating},
    \citet{lacroix2009understanding}, \citet{lim2010pediluma},
    \citet{mutsuddi2012text}, \citet{ploderer2008hey},
    \citet{young2010twitter} -.
  \item
    Éducation, apprentissage - \textcolor{red}{à lire}
    \citet{berque2011design}, \citet{chang2008playful},
    \citet{goh2012impact}, \citet{lucero2006persuasive},
    \citet{reis2011perception} -.
  \item
    Économie, commerce, marketing - \textcolor{red}{à lire}
    \citet{cugelman2008website}, \citet{russell2008benevolence} -.
  \item
    sécurité, sûreté - \textcolor{red}{à lire}
    \citet{bergmans2013reducing}, \citet{chittaro2012passengers},
    \citet{hartwig2013safety}, \citet{miranda2013examining} -.
  \item
    Divertissement - \textcolor{red}{à lire}
    \citet{centieiro2012applaud}, \citet{reitberger2012persuasive} -.
  \item
    Consommation et/ou comportement écologique - \textcolor{red}{à lire}
    \citet{centieiro2011location}, \citet{ruijten2012bridging} -.
  \item
    Gestion de l'énergie - \citet{ham2010ambient},
    \citet{medland2010curbing} , \citet{rodgers2011exploring},
    \citet{gamberini2012tailoring}, \citet{costanza2012understanding},
    \citet{weiss2009handy}, \citet{pereira2013understanding},
    \citet{elsmore2010neighbourhood}, \citet{petkov2012personalised},
    \citet{kjeldskov2012using}, \citet{paay2014design},
    \textcolor{red}{à lire} \citet{filonik2013customisable},
    \citet{foster2010wattsup}, \citet{gamberini2012tailoring},
    \citet{kim2010design}, \citet{roubroeks2010dominant},
    \citet{ruijten2012bridging}, \citet{ruijten2011unconscious},
    \citet{valkanova2013reveal} -.
  \end{itemize}
\item
  Modèle de persuasion de \citet{kaptein2010persuasion} composé du
  modèle de la probabilité d'élaboration de
  \citet{petty1986elaboration}, de la théorie Motivation, Opportunité,
  Capacité de \citet{maclnnis1989information}, de la théorie du
  comportement planifié de \citet{dillon1996user}, du conditionnement
  classique de \citet{patterson1987rabbit} et du conditionnement opérant
  de \citet{skinner1976behaviorism}.
\item
  Modèle de conduite du changement comportemental de
  \citet{prochaska2005transtheoretical}.
\item
  Principes de persuasion de \citet{negri2015ludo} inspirés par les
  principes de persuasion de \citet{fogg2002persuasive},
  \citet{oinas2009persuasive}, \citet{nemery2012development},
  \citet{cialdini2004influence} et les principes de ludification de
  \citet{zichermann2011gamification}.
\item
  Espace de classification de \citet{cano2015persuasive}.
\end{enumerate}

\subsection{Interface utilisateur
tangible}\label{interface-utilisateur-tangible}

\begin{enumerate}
\def\labelenumi{\arabic{enumi}.}
\item
  Définition des TUIs - \citet{wellner1993back},
  \citet{fitzmaurice1995bricks}, \citet{ishii1997tangible},
  \citet{ullmer2000emerging}, \citet{shaer2010tangible} -.
\item
  Applications et exemples :

  \begin{itemize}
  \itemsep1pt\parskip0pt\parsep0pt
  \item
    Communication sociale - \citet{werner2008unitedpulse},
    \citet{ernevi2005interactive}, \citet{chang2001lumitouch} -.
  \item
    Apprentissage - \citet{zufferey2009tinkerSheets},
    \citet{underkoffler1998illuminatinglight}, \citet{raffle2004topobo},
    \citet{frei2000curlybot} -.
  \item
    Divertissement et éducation - \citet{zigelbaum2007tangible},
    \citet{ryokai2004iobrush}, \citet{frey2014teegi},
    \citet{gervais2015tobe} -.
  \item
    Musique et Performance - \citet{jorda2007reactable},
    \citet{schiettecatte2008audiocubes}, \citet{patten2002audiopad},
    \citet{newton2003block} -.
  \item
    Planification et résolution de problème - \citet{ishii2008tangible},
    \citet{underkoffler1999urp}, \citet{patten2007mechanical},
    \citet{jacob2002tangible} -.
  \item
    Programmation - \citet{suzuki1995interaction},
    \citet{horn2008tangible}, -.
  \item
    Visualisation d'information - \citet{couture2008geotui},
    \citet{hinckley1994passive} -.
  \end{itemize}
\end{enumerate}

\subsection{Technologie persuasive
ambiante}\label{technologie-persuasive-ambiante}

\begin{enumerate}
\def\labelenumi{\arabic{enumi}.}
\item
  Définition de la technologie persuasive ambiante par
  \citet{davis2008towards}, \citet{ham2010ambient}.
\item
  Applications et exemples :

  \begin{itemize}
  \itemsep1pt\parskip0pt\parsep0pt
  \item
    Santé et exercices - \citet{faber2011aulura}, \citet{kim2010inair},
    \citet{nakajima2008reflecting} -
  \item
    Économie, commerce, marketing, sécurité, sûreté -
    \citet{kalnikaite2011nudge} -.
  \item
    Éducation, apprentissage - \citet{reis2011perception} -.
  \item
    Gestion de l'énergie - \citet{belley2006semaphore},
    \citet{belley2006coupe}, \citet{evans2009artful},
    \citet{gustafsson2005power}, \citet{kyoto2005wattson},
    \citet{jonsson2010watt}, \citet{ernevi2005energy},
    \citet{gyllensward2006visualizing}, \citet{lagerkvist2016flower},
    \citet{lagerkvist2016disappearing}, \textcolor{red}{à lire}
    \citet{rogers2010ambient}, \citet{kuznetsov2010upstream},
    \citet{valkanova2013reveal} -.
  \end{itemize}
\end{enumerate}

\newpage
\bibliographystyle{apalike-fr}
\bibliography{resources/biblio}

\end{document}

