%% Example of a LaTeX source file for a COLING-2012 submission
%% last updated: July 10, 2012
%% Optional instructions for authors within the tex file are provided as comments and start with 'for authors:...'
\documentclass[10pt,a5paper,twoside]{article}
\usepackage{coling2012}
\usepackage[frenchb]{babel}
\title{Interfaces Tangibles comme Aide à la Maîtrise de l'Énergie}
%for authors: in case of more than four author names ref. to commented line below 
%\author{$Annie~SMITH^{1, 2}~~~LI~Xiao Dong^{1, 3}$\\$~~~Third~Author^{1, 2}~~~Fourth~Author^{1, 3}~~~ Fifth~Author^{2, 3}$\\
\author{$Maxime~DANIEL^{1, 2}~Guillaume~RIVIERE^{1, 2}~Nadine~COUTURE^{1, 2}$\\
{\small  	(1) Université de Bordeaux, Bordeaux, France\\ 
 		(2) Estia, Bidart, France\\
  \texttt{m.daniel@estia.fr, g.riviere@estia.fr, n.couture@estia.fr} \\ 
}}

\begin{document}
\maketitle
%% The first mandatory ABSTRACT (\abstractEn) section below is for the English language
%Example for English + optional language keywords list
\keywordsEn{IHM, TUI, Technologie persuasive, Gestion de l'énergie, Espace social physique}
\section{Context}
Energy transition is an important issue today and it is a challenge accepted by many governments, regions and companies. Beyond the augmentation of energy production efficiency, affecting consumption curve is a second way studied in the future context of European SmartGrid or autonomous local micro-networks. 
But energy is an invisible and intangible entity which make people less aware of its presence. Therefore, people tend to dispose of energy without regards to quantity, time and duration.\par
In the last decade of Human-Computer Interaction (HCI) Research, many works were inspired by Persuasive technology \cite{fogg2002persuasive} in order to persuade individuals to change their consummation behaviour using energy bill reduction as motivation artifact. Other artifacts are studied today like social pressure by putting people into competition using Gamification \cite{deterding2011game} or even Serious Game \cite{abt1987serious}. \par
Tangible User Interfaces (TUI) are systems using physical objects to represent and manipulate digital information \cite{ishii1997tangible}. TUIs show many abilities and one of them is the ability to strongly embody abstract entities into physical objects \cite{shaer2010tangible}.
\section{Objectives}
We wish to explore how TUIs can embody energy to allow user interactions with energy in our physical world and thus get users closer to energy \cite{pierce2010materializing}\cite{gustafsson2005power}. Persuasion is also a subject of interest, we want to study how TUIs as persuasive technologies can sensibilise users to energy management problematics (consummation, production, storage) in physical social spaces (e.g., a school, a company, a public place) where users are not motivated by energy bill reduction. To go further into persuasion, we also wish to explore how TUIs can support behaviour change and assist durable behaviour creation.

%%-------------
%for authors: if only English language option is chosen comment the \abstractOL section above and use \keywordsEn below 
%for authors: else use add title and abstract to \abstractOL section above and use \keywordsOL below (case-sensitive commands)

%for authors: for keywords section either use \keywordsEn OR \keywordsOL below as relevant
%Example for English only keywords list
%\keywordsEn{Here a list of keywords in English}


%%--------------

\newpage
%'apalike-fr' style below applies smallcaps style on author names
%in order to apply 'apalike-fr' the babel package must be given [frenchb] option instead of [english]
% \usepackage[frenchb]{babel} also causes title "References" to render with French accents like "R\'ef\'erences"
\bibliographystyle{apalike-fr}

%'apa' style does not apply "smallcaps style" on author names and goes with the [english] option in the babel package

%\bibliographystyle{apa}

\bibliography{colingbiblio}
\nocite{TALN2007,LaigneletRioult09,LanglaisPatry07,au1972,cks1981,mb2012}

%%================================================================
\end{document}
